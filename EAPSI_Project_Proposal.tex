\documentclass[a4paper,10pt]{article}
%\usepackage[utf8x]{inputenc}
\usepackage{amsmath}
\usepackage{amsfonts}
\usepackage{natbib}
\usepackage{fullpage}

%opening
\title{A Structural Estimation of Chinese Conspicuous Consumption}
\author{David Jinkins}

\begin{document}

\maketitle

\section{Introduction}
Economists have long theorized about the role of conspicuous consumption in household spending decisions.  When a consumer decides which products to buy, part of her decision is based on what society will infer about her after observing what she chooses.  Introspection lends credence to this theory, but researchers have only recently attempted to measure the importance of conspicuous consumption in consumer demand.  I am currently developing a structural model of conspicuous consumption and consumer demand which can be estimated using household budget survey data.  During my eight weeks in China, I plan on using household expenditure data to estimate the importance of conspicuous consumption in Chinese demand.  This estimate can then be compared to my benchmark American estimate as a first step in understanding the relationship between conspicuous consumption and culture.

In this research agenda, I take conspicuous consumption to be a mechanism for signaling wealth.\footnote{It is important to note that this is different than signaling social status, to the extent that status is not perfectly correlated with wealth.  Also, the signal is not about relative wealth, but about absolute wealth, which differentiates this research from the ``positional goods'' literature associated with Cornell's Robert Frank.}  I am taking the point of view that the reason people consume conspicuous goods is that they can be at least partially observed by society, and inform society about the consumer's (unobservable) wealth.  There have been several recent empirical studies in this vein of research.  In one notable study, \citet{Blochetal2004} estimate a reduced form model of dowries and spending on wedding celebrations in India, and find results consistent with the idea that dowries buy the quality of husband (invisible), and weddings signal the status of husbands to society (conspicuous).  In a forthcoming research paper, \citet{Heffetz2011} conducts a telephone survey in the United States to determine the visibility of consumption goods.  Heffetz then analyzes household budget survey data, and finds evidence that the relitively visible goods identified by the survey are being used as a means to signal wealth. 

Since conspicuous consumption is a distortion from what would otherwise be an individual's optimal choice of goods, it is costly in terms of consumer welfare.  If people care deeply about social status, then the welfare cost of the distortion may be large.  Moreover, if everyone's wealth was known to society, then there would be no reason to overconsume conspicuous goods.  This point seems particularly poignant when considering the spending habits of the very poor.  A recent study reports that in parts of India, the \emph{median} household making under a dollar a day spends 10\% of its income on festivals--this while 43\% of such households did not have enough to eat throughout the year \citep{BanerjeeDuflo2007}.  Could it be that the poor overconsume festivals as a signal that they are not even poorer?  If so, there are policies such as taxing festival spending that may substantially increase welfare.  Measuring the importance of conspicuous consumption to consumers is a first step in gauging the social benefit of such a policy.

It is worthwhile to take a moment to differentiate the estimation of conspicuous consumption at hand from the relatively large literature on estimating interdependent preferences.  Beginning with James Duesenberry's 1949 doctoral thesis,\footnote{Later published as \citep{Duesenberry1949}} researchers have theorized that the consumption of neighbors affects own demand.  A typical econometric model in this literature lets household demand parameters depend linearly on the (weighted) average consumption of a reference group. A relationship between neighbor consumption and own consumption is taken to mean that preferences are interdependent.  The literature, however, does not take a stance on why consumption is interdependent, and it is hard to interpret what the estimated coefficients really mean.\footnote{To cite one of the first econometric papers in this literature, \citet{Pollak1976} speculates that exposure to the consumption of ``superior'' goods leads one to develop a taste for them.}  

Estimating my model on Chinese data and comparing to the American estimates will be particularly illuminating given the anecdotal literature on conspicuous consumption in China.  There have been numerous articles in the popular press in recent years describing Chinese luxery fever.  One entertaining example describes a caravan of 30 black limosines sent to Xian's airport to pick up a dog purchased for 1.5 million dollars.\footnote{http://www.telegraph.co.uk/news/worldnews/asia/china/8383084/1-million-for-worlds-most-expensive-dog.html}  In addition, at least one scholarly survey finds that Chinese value conspicuous consumption more than Americans \citep{Podoshenetal2010}.  My research in China next summer will be able to test this hypothesis in a structural model.  I will conduct a statistical test comparing American and Chinese utility weights on conspicuous consumption.          

\section{Intellectual Merit}
The model presented in this proposal has several contributions to the literature.  The model is, to my knowledge, the first to allow structural estimation of the weight of conspicuous consumption in a consumer's utility function.  It is the first model to employ actual data identifying the visibility of different good categories to structurally identify conspicuous consumption.  The model takes the signaling nature of conspicuous consumption seriously so that consumer's spending decisions depend on social beliefs as well as the equilibrium spending decisions of all other members of society.  Finally, my research in China will allow me to structurally and statistically compare Chinese and American conspicuous consumption.        

\section{Model}
Consider a continuum of agents of wealth type $w\in[\underbar{w},\overline{w}]$ with distribution $h$.  Let agents have access to goods in the set $I$, $\|I\|<\infty$, with exogenously given prices $\{p_i\}_I$.  In this set up, agents first buy goods, and then society observes a noisy signal of each agent's purchased bundle.  Let the optimal consumption bundle for an agent of type $w$ be given by the vector valued function $g:[\underbar{w},\overline{w}]\rightarrow \mathbb{R}_+^{\|I\|}$, $g\in\Gamma$. Let an agent's fundamental utility, i.e. the utility he gets directly from consuming a bundle of goods, be given by $u:\mathbb{R}_+^{\|I\|}\rightarrow \mathbb{R}$.  Let $\alpha$ be the weight of conspicuous consumption, and let $B:\mathbb{R}_+^{\|I\|}\times \Gamma \rightarrow [\underbar{w},\overline{w}]$ give society's expectation of agents utility based on the noisy observation.  Note that all the action here is on the side of the agent--society just forms a belief in a mechanical way.  Suppose that society observes agents true consumption of good $i\in I$ plus $\epsilon_i \sim N(0,\sigma_i^2)$.  Denoting good $i$'s probability distribution function by $f_i$, we can now write societies belief function as:
\begin{equation}
 B(X,g) = \int u(g(w)) \Pi_i f_i(X_i - g_i(w)) h(w) dw, 
\end{equation}
and an agents total utility function by:
\[ 
 U(C) = (1-\alpha) u(C) + \alpha \int B(C+\epsilon,g) f(\epsilon) d\epsilon.
\]
This expression can by further simplified using properties of the normal distribution:
\begin{equation}
 U(C) = (1-\alpha) u(C) + \alpha \int u(g(w)) h(w) \Pi_i f_i^*(C_i-g_i(w)) dw,
\end{equation}
where $f_i^*$ is the normal probability density function with mean zero and variance $2\sigma_i^2$.

The tentative estimation strategy is to get $h$ and $g$ non-parametrically from the household budget survey, get the variances on the noise from the survey conducted by Ori Heffetz, and get prices from some other government publication (in the United States, NIPA).  After choosing some parametric form for utility, use maximum likelihood or GMM to find the parameters which make model based optimal consumption choice closest to the equilibrium choices $g$ seen in the data.  I am currently working on choosing the best place to put the econometric error.  This is still very much a work in progress. (NOTE: Hopefully by the time I submit this, I will be able to flesh this section out a bit more.)     

\section{Data}
There are several datasets which I may use for estimation depending on availability as I move forward.  Ideally I would use (some portion of) the Rural/Urban Household Income and Expenditure Survey which has been collected since 1980 by the Chinese National Bureau of Statistics.  This data has been used widely by researchers both inside and outside of China--notably Shaohua Chen and Martin Ravallion at the World Bank.  In a recent article on obtaining household survey data in China, \citet{GustafssonShi2006} make the point that the government process of collecting household surveys is quite decentralized, so it is often useful to contact regional statistical agencies.\footnote{It is worth remarking that two papers by Yaohui Zhao, my host at Peking University's China Center for Economic Studies, are mentioned in the article.}  If access to this data is not feasible, there is a free dataset available from the University of Michigan called the Chinese Household Income Project.  This is a survey of rural and urban households conducted over the course of several years in the late 1980s.  The focus of this project was sources of rural household income, but household expenditure in a number of categories was also recorded.\footnote{Categories included are: Food, Fuel, Medical and Health Products, Education, Transportation, Postal and Telecommunications, and Electricity.}  Although more consumption categories would be better, the categories included in this project may be enough for me to estimate my model if need be.  To estimate the model, I will also need corresponding price data, but at this level of aggregation I should be able to find what I need in published handbooks of the National Bureau of Statistics.    

\section{Host Institution}
The China Center for Economic Research (CCER) at Peking University is the premiere location in China for studying the Chinese economy.  Justin Lin, now the chief economist at the World Bank, was the director of CCER from its inception in 1994 until 2008.  There are currently 33 full time faculty listed on the website.  My host, Professor Yaohui Zhao, received her PHD in economics from the University of Chicago in 1995. She has published articles in such journals as the American Economic Review and the Chinese Economic Review.  Professor Zhao is an expert in empirical labor economics and the analysis of rural Chinese households, especially in the analysis of Chinese labor migration.  She has also published peer-reviewed articles in the fields of health and education economics. Professor Zhao's advice regarding the acquisition and analysis of household data will be a valuable resource for me.  In addition to the resources available at CCER, Peking University hosts both an economics department and the Guanghua School of Business.  Three researchers at Guanghua recently published an empirical paper on income and consumption inequality based on the Urban Household Income and Expenditure Survey mentioned above \citep{Caietal2010}.  

\section{Qualifications and Expectations}
I can read, write, and fluently speak Mandarin Chinese.  I am skilled at coding and data manipulation in MATLAB and STATA.  I have experience working with several large datasets, including individual level Taiwanese health care data and transactions level Colombian trade data.  I have lived in Beijing and Taiwan, and have worked in an academic setting with Taiwanese researchers.  My experience and skills make me well suited to conducting empirical research in China.  

I hope that relationships I build next summer will lead to long-term collaboration with researchers at Peking University.  My skill in the Chinese language and experience working with Taiwanese academics puts me in an ideal position to explore Chinese data in future research.  In addition to signaling and consumer demand, I am interested in the literature on rural-urban (and international) migration, as well as misallocation of production inputs.  China's growth over the last quarter century makes it a well-suited location to study these topics. The relationships I build with Chinese academics will allow me to benefit from their expertise regarding the Chinese economy and facilitate future access to Chinese domestic data. 

\section{Letters of Recommendation}
My letters of recommendation will be written by Professor Jim Tybout and Professor Jonathan Eaton. Both professors are faculty at the Penn State University Economics Department.  Professor Tybout is the adviser for my third-year paper, and also taught one of my development field courses.  Professor Eaton taught one of my trade field courses.  In addition, I am working as a research assistant on a joint research project of Professor Eaton and Professor Tybout's regarding search and learning in international trade.

\section{Broader Impacts}
Measuring conspicuous consumption is important for several reasons.  Conspicuous consumption is a wasteful signal.  If my research indicates that conspicuous consumption is an important determinant of consumer demand, it may be efficient for governments to implement taxes on the conspicuous goods identified by the Heffetz visibility survey, or a more detailed survey not yet conducted.  Historically governments have done something close to this with so-called luxury taxes and sumptuary laws.  Alternatively (and less plausibly), individual wealth could be publicly disclosed by the government.  If my estimates for China and the United States are significantly different, there may be a cultural aspect to conspicuous consumption.  Again, this may inform government policy in terms of taxation or even education.

Estimating conspicuous consumption in a developing country may also provide insight into the strange consumption habits of the very poor.  \citet{BanerjeeDuflo2007} describe some of the poorest of the poor in India spending large sums of their household budgets on festivals, alcohol, and cigarettes.  If my estimates of Chinese conspicuous consumption turn out to be large, it lends some support to the hypothesis that the poor are overconsuming relatively visible festival and recreational spending as a signal of their wealth.

      

\bibliographystyle{plainnat.bst}
\bibliography{biglist.bib}
\end{document}
